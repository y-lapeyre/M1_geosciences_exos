\chapter{Equations différentielles}
\begin{enumerate}
\item Résoudre $y^{\prime}+y=e^{-x}$.
\item Résoudre $3 x y^{\prime}-4 y=x$.
\item Résoudre $y^{\prime}-y^2=0$.
\item $x\left(x^2-1\right) y^{\prime}+2 y=x^2$ : montrer que cette équation admet une seule solution sur $\mathbb{R}$. Indication : on pourra chercher des réels $a, b, c$ tels que $\frac{1}{x\left(x^2-1\right)}=\frac{a}{x}+\frac{b}{x+1}+\frac{c}{x-1}$.
\item $x y^{\prime}+2 y=\cos x$ : existe-t-il des solutions sur $\mathbb{R}$ ?
\item Résoudre $y^{\prime} \cos (x)+y \sin (x)+y^3=0$ avec le changement de variable $u=1 / y^2$.
\item $(x+1) y^{\prime}-2 y=(x+1)^4$ : Déterminez l'ensemble des solutions sur $\mathbb{R}$ de cette équation. Déterminez la ou les solutions passant par le point $(0,0)$ et celle(s) passant les points $(0,0)$ et $(-2,1)$.
\item Résoudre $y^{\prime \prime}+5 y^{\prime}+4 y=\left(x^2+2 x\right) e^{-x}+3 e^{2 x}-4$.
\item Résoudre $4 y^{\prime \prime}-2 y^{\prime}=x-1$.
\item Résoudre $y^{\prime \prime}+2 y^{\prime}+y=\sin ^2 x$.
\item Résoudre $y^{\prime \prime}-2 y^{\prime}+y=e^x \sin x$.
\item Résoudre $y^{\prime \prime}-2 y^{\prime}+y=\frac{e^x}{\cos ^2 x}$.
\item Résoudre $y^{\prime \prime}-4 y^{\prime}+4=\left(x^2+1\right) e^{2 x}$.
\item $x^2 y^{\prime \prime}-2 x y^{\prime}+\left(2-x^2\right) y=0$. En faisant le changement de fonction $y=z x$ ( $z$ devient la nouvelle fonction inconnue), déterminer l'ensemble des solutions sur $\mathbb{R}^{+*}$ et $\mathbb{R}^{-*}$. Existe-t-il des solutions sur $\mathbb{R}$ ?
\item Résoudre l'équation différentielle
$$
y^{\prime \prime}-y^{\prime}+y=0
$$
On veut ensuite déterminer toutes les fonctions $f: \mathbb{R}^{+*} \rightarrow \mathbb{R}$ dérivables telles qu'elles vérifient l'équation
$$
\forall x \in \mathbb{R}^{+*}, f^{\prime}(x)=f\left(\frac{1}{x}\right)
$$
On pose $g(t)=f\left(e^t\right)$, montrer que g est solution de la première équation. Trouver ensuite toutes les fonctions f vérifiant la seconde.
\end{enumerate}
