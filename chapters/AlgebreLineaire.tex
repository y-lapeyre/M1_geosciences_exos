\chapter{Algèbre linéaire}


\section*{Exercice 1: Produit matriciel}

Calculer le produit AB :
\[
\begin{gathered}
A=\left(\begin{array}{ccc}
1 & 2 & 5 \\
1 & 0 & -1 \\
3 & -1 & 1 \\
1 & 1 & 0
\end{array}\right), \text { et } B=\left(\begin{array}{cc}
3 & -7 \\
4 & 0 \\
1 & 1
\end{array}\right) ; \\
A=\left(\begin{array}{lll}
1 & 1 & 2 \\
3 & 1 & 0
\end{array}\right), \text { et } B=\left(\begin{array}{lll}
3 & 0 & 2 \\
4 & 0 & 1 \\
1 & 1 & 2
\end{array}\right) ;
\end{gathered}
\]

Calculer \(A^{0}, A^{1}, A^{2}, A^{3}\) avec
\[
A=\left(\begin{array}{ccc}
1 & 0 & -2 \\
2 & 0 & 0 \\
0 & 1 & 3
\end{array}\right)
\]

\section*{Exercice 2: Matrices d'endomorphismes particuliers}

Donner les matrices associées aux endomorphismes réalisant :
- Dans le plan \(\mathbb{R}^{2}\), une rotation d'angle \(\pi / 4\).
- Dans le plan \(\mathbb{R}^{2}\), une symétrie par rapport à la droite dirigée par le vecteurs de coordonnées \((1,1)\).
- Dans \(\mathbb{R}^{3}\), la projection sur la droite dirigée par le vecteur \(e_{1}=(1,0,0)\).

On se placera chaque fois dans les bases canoniques des espaces mentionnés.

\section*{Exercice 3: Inverse de matrices}

Vérifier que \(B\) est l'inverse de \(A\).
\[
A=\left(\begin{array}{ccc}
1 & 0 & -2 \\
2 & 4 & 1 \\
-1 & 1 & 3
\end{array}\right) \text {, et } B=\left(\begin{array}{ccc}
-11 & 2 & -8 \\
7 & -1 & 5 \\
-6 & 1 & -4
\end{array}\right)
\]

Calculer l'inverse, s'ils existent, de
\[
\begin{aligned}
A & =\left(\begin{array}{cc}
2 & 4 \\
-1 & 3
\end{array}\right) \\
B & =\left(\begin{array}{lll}
1 & 0 & 1 \\
0 & 1 & 0 \\
1 & 0 & 1
\end{array}\right) \\
C & =\left(\begin{array}{lll}
2 & 3 & 4 \\
1 & 3 & 2 \\
3 & 1 & 1
\end{array}\right) \\
D & =\left(\begin{array}{lll}
2 & 0 & 4 \\
1 & 0 & 2 \\
1 & 1 & 1
\end{array}\right)
\end{aligned}
\]

\section*{Exercice 4: Systèmes linéaires}

Résoudre les systèmes suivants (que l'on pourra ou non écrire sous forme matricielle) :
\[
\begin{gathered}
\left\{\begin{array}{cr}
x & -2 z=-2 \\
2 x+4 y+z= & 3 \\
-x+y+3 z= & 1
\end{array}\right. \\
\left\{\begin{array}{ccc}
2 x-y+3 z+2 t & =-2 \\
-3 x+y-5 z+t & =-1 \\
5 x-5 y+10 z+4 t & =-1 \\
-4 x+2 y-7 z-3 t= & 0
\end{array}\right.
\end{gathered}
\]

Résoudre les systèmes \(A X=B\), avec :
\[
\begin{aligned}
& A=\left(\begin{array}{lll}
1 & 0 & 1 \\
0 & 1 & 0 \\
1 & 0 & 1
\end{array}\right), \text { et } B=\left(\begin{array}{l}
1 \\
0 \\
1
\end{array}\right) \\
& A=\left(\begin{array}{lll}
1 & 1 & 1 \\
3 & 1 & 0 \\
2 & 1 & 1
\end{array}\right), \text { et } B=\left(\begin{array}{l}
2 \\
1 \\
1
\end{array}\right)
\end{aligned}
\]

\section*{Solutions}

\section*{Produit matriciel}
\[
\begin{gathered}
A B=\left(\begin{array}{cc}
16 & -2 \\
2 & -8 \\
6 & -20 \\
7 & -7
\end{array}\right) \\
A B=\left(\begin{array}{ccc}
9 & 2 & 7 \\
13 & 0 & 7
\end{array}\right) \\
A^{2}=\left(\begin{array}{ccc}
1 & -2 & -8 \\
2 & 0 & -4 \\
2 & 3 & 9
\end{array}\right) \\
A^{3}=\left(\begin{array}{ccc}
-3 & -8 & -26 \\
2 & -4 & -16 \\
8 & 3 & 23
\end{array}\right)
\end{gathered}
\]

Inverse de matrices
\[
A^{-1}=\frac{1}{10}\left(\begin{array}{cc}
3 & -4 \\
1 & 2
\end{array}\right)
\]
\(B\) n'est pas inversible !
\[
C^{-1}=-\frac{1}{15}\left(\begin{array}{ccc}
1 & 1 & -6 \\
5 & -10 & 0 \\
-8 & 7 & 3
\end{array}\right)
\]

D n'est pas inversible !

\section*{Matrices d'endomorphismes particuliers}
- Rotation de \(\pi / 4: \frac{\sqrt{2}}{2}\left(\begin{array}{cc}1 & -1 \\ 1 & 1\end{array}\right)\)
- Symétrie par rapport à la droite dirigée par le vecteur \((1,1):\left(\begin{array}{ll}0 & 1 \\ 1 & 0\end{array}\right)\)
- Projection sur la droite dirigée par \((1,0,0):\left(\begin{array}{lll}1 & 0 & 0 \\ 0 & 0 & 0 \\ 0 & 0 & 0\end{array}\right)\)

\section*{Systèmes linéaires}
- Solution premier système : \(\left(\begin{array}{c}20 \\ -12 \\ 11\end{array}\right)\).
- Deuxième système (faire \(L_{4} \leftarrow \mathrm{~L}_{4}+2 L_{1}\) ) puis continuer... On trouve : A FINIR
- Troisième système : la matrice n'est pas inversible. Les solutions sont les vecteurs : \((x, 0,-x), x \in \mathbb{R}\).
- Quatrième système : la matrice est inversible et a pour déterminant -1 . On calcule son inverse et on trouve : \(\left(\begin{array}{ccc}-1 & 0 & 1 \\ 3 & 1 & -3 \\ -1 & -1 & -2\end{array}\right)\).
La solution est donc : \(\left(\begin{array}{c}-1 \\ 4 \\ -1\end{array}\right)\).