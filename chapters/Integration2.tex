\chapter{Intégration 2: les champs}

\section*{Exercice 1: Définition et dérivées partielles}
\begin{enumerate}
    \item  déterminer le domaine de définition des fonctions:
    \begin{enumerate}[a)]
        \item $f(x, y)=\frac{\sqrt{y+x^2}}{\sqrt{y}}$
        \item $f(x, y)=\frac{\ln (y)}{\sqrt{x-y}}$
        \item $f(x, y)=\ln (x+y)$
    \end{enumerate}
    \item  La loi de Boyle Mariotte, pour une mole de gaz parfait, donne:
    $$
    P V=R T
    $$
    avec $P$ la pression du gaz, $V$ son volume, $R$ la constante des gaz parfaits et $T$ la température.
    \begin{enumerate}[a)]
        \item Calculer $\frac{\partial P}{\partial T}$ et $\frac{\partial P}{\partial V}$
        \item Calculer ces deux dérivées partielles si l'on considère la relation de Van der Waals:
        $$
        \left(P+\frac{a}{V^2}\right)(V-b)=R T
        $$
        avec $a$ et $b$ deux réels.
    \end{enumerate}
    \item Calculer les dérivées partielles d'ordre 1 des fonctions:
    \begin{enumerate}[a)]
        \item $f(x, y)=y^5-3 x y$
        \item $f(x, y)=x^2+3 x y^2-6 y^5$
        \item $f(x, y)=x \cos \left(e^{x y}\right)$
        \item $f(x, y)=\frac{x}{y}$
    \end{enumerate}
\end{enumerate}

\section*{Exercice 2: Differentielles}

\begin{enumerate}
    \item Sachant que la fonction $f: \mathbb{R}^2 \rightarrow \mathbb{R}$ est différentiable et que: $f(2,5)=6$, $\partial_x f(2,5)=1, \partial_y f(2,5)=-1$, donner la valeur approchée de $f(2.2,4.9)$.
    \item Calculer les différentielles des fonctions suivantes:
    \begin{enumerate}[a)]
        \item $f(x, y)=\sqrt{x^2+3 y^2}$
        \item $R=\alpha \beta^2 \cos (\gamma)$
        \item $T=\frac{v}{1+u v w}$
    \end{enumerate}
    \item on mesure un rectangle et on obtient une largeur de 30 cm et une longueur de 24 cm , avec une erreur d'au plus 0.1 cm pour chaque mesure. Estimer l'aire du rectangle.
\end{enumerate}

\section*{Exercice 3: Courbes de niveau et gradient}

\begin{enumerate}
    \item Déterminer les courbes de niveau de la fonction $f(x, y)=x+y-1$ et les représenter sur un graphe.
    \item Calculer le gradient de: $f(x, y, z)=x^2+y^2-\ln (x y z)$
\end{enumerate}

\section*{Exercice 4: Calculs d'extrema}
déterminer les extrema locaux des applications suivantes et donner le maximum de précisions possibles sur ces extrêma:

\begin{enumerate}
    \item $f:(x, y) \mapsto x^4+x^2+y^2+y^4$
    \item $f:(x, y) \mapsto x^3+2 x y-5 x+5 y$
    \item $h:(x, y) \mapsto(3 x+7) e^{-\left((x+1)^2+y^2\right)}$
\end{enumerate}

\section*{Exercice 5: Intégrales de fonctions de plusieurs variables}
1. Soit $D$ le domaine:

$$
D=\left\{(x, y) \in \mathbb{R}^2 \mid x \geq 0, y \geq 0, x+y \leq 1\right\}
$$


Calculer $\iint_D f(x, y) d x d y$ pour
\begin{enumerate}[a)]
    \item  $f(x, y)=x^2+y^2$
    \item $f(x, y)=x y(x+y)$
\end{enumerate}


2. Calculer l'intégrale double $\iint_D f(x, y) d x d y$ pour:

(a) $f(x, y)=x$ et $D=\left\{(x, y) \in \mathbb{R}^2 \mid y \geq 0, x-y+1 \geq 0, x+2 y-4 \leq 0\right\}$

(b) $f(x, y)=x+y$ et $D=\left\{(x, y) \in \mathbb{R}^2 \mid 0 \leq x \leq 1, x^2 \leq y \leq x\right\}$

3. soit $D$ le domaine: $D=\left\{(x, y) \in \mathbb{R}^2 \mid-1 \leq x \leq 1, x^2 \leq y \leq 4-x^3\right\}$. Calculer l'aire de $D$

4. soit $D$ le domaine: $D=\left\{(x, y, z) \in \mathbb{R}^3 \mid x>0, y>0, z>0, x+y+z<1\right\}$. Calculer $\iiint_D(x+y+z)^2 d x d y d z$

5. calculer le volume d'une partie de la sphère de centre $(0,0,0)$ et de rayon $R$, comprise entre les plans d'équation $z=h_1$ et $z=h_2$ (sachant que $R$ vérifie $R \geq h_1>h_2 \geq-R$ ).

\section*{Exercice 6: coordonnées cylindriques et sphériques}


Soit $D$ le domaine: $D=\left\{(x, y) \in \mathbb{R}^2 \mid x^2+y^2-2 x \leq 0\right\}$.

(a) Exprimer $D$ en coordonnées polaires

(b) Calculer $\iint_D \sqrt{x^2+y^2} d x d y$


\section*{Correction 1: Définition et dérivées partielles}

\begin{enumerate}
    \item Domaines de définition:
    \begin{enumerate}[a)]
        \item $\mathcal{D}_f = \{(x, y) \in \mathbb{R}^2 \mid y > 0 \text{ et } y + x^2 \geq 0\}$
        \item $\mathcal{D}_f = \{(x, y) \in \mathbb{R}^2 \mid y > 0 \text{ et } x - y > 0\}$
        \item $\mathcal{D}_f = \{(x, y) \in \mathbb{R}^2 \mid x + y > 0\}$
    \end{enumerate}

    \item Loi des gaz parfaits:
    \begin{enumerate}[a)]
        \item $P = \frac{RT}{V}$ donc:
        $\frac{\partial P}{\partial T} = \frac{R}{V}$, $\frac{\partial P}{\partial V} = -\frac{RT}{V^2}$
        
        \item Van der Waals: $P = \frac{RT}{V-b} - \frac{a}{V^2}$ donc:
        $\frac{\partial P}{\partial T} = \frac{R}{V-b}$, $\frac{\partial P}{\partial V} = -\frac{RT}{(V-b)^2} + \frac{2a}{V^3}$
    \end{enumerate}

    \item Dérivées partielles:
    \begin{enumerate}[a)]
        \item $\frac{\partial f}{\partial x} = -3y$, $\frac{\partial f}{\partial y} = 5y^4 - 3x$
        \item $\frac{\partial f}{\partial x} = 2x + 3y^2$, $\frac{\partial f}{\partial y} = 6xy - 30y^4$
        \item $\frac{\partial f}{\partial x} = \cos(e^{xy}) - xye^{xy}\sin(e^{xy})$, $\frac{\partial f}{\partial y} = -x^2e^{xy}\sin(e^{xy})$
        \item $\frac{\partial f}{\partial x} = \frac{1}{y}$, $\frac{\partial f}{\partial y} = -\frac{x}{y^2}$
    \end{enumerate}
\end{enumerate}

\section*{Correction 2: Différentielles}

\begin{enumerate}
    \item $f(2.2,4.9) \approx f(2,5) + \partial_x f(2,5)\cdot 0.2 + \partial_y f(2,5)\cdot (-0.1) = 6 + 0.2 - (-0.1) = 6.3$
    
    \item Différentielles:
    \begin{enumerate}[a)]
        \item $df = \frac{x}{\sqrt{x^2+3y^2}}dx + \frac{3y}{\sqrt{x^2+3y^2}}dy$
        \item $dR = \beta^2\cos(\gamma)d\alpha + 2\alpha\beta\cos(\gamma)d\beta - \alpha\beta^2\sin(\gamma)d\gamma$
        \item $dT = \frac{vw}{(1+uvw)^2}du + \frac{1}{(1+uvw)^2}dv - \frac{uv^2}{(1+uvw)^2}dw$
    \end{enumerate}
    
    \item $A = L\times l = 30\times 24 = 720$ cm²\\
    $dA = \frac{\partial A}{\partial L}dL + \frac{\partial A}{\partial l}dl = ldL + Ldl$\\
    Erreur maximale: $\Delta A \approx |24\times 0.1| + |30\times 0.1| = 5.4$ cm²\\
    Aire: $720 \pm 5.4$ cm²
\end{enumerate}

\section*{Correction 3: Courbes de niveau et gradient}

\begin{enumerate}
    \item Courbes de niveau: $x+y-1 = k$ soit $y = 1 - x + k$\\
    Famille de droites parallèles de pente $-1$
    
    \item $\nabla f(x,y,z) = \left(2x - \frac{1}{x}, 2y - \frac{1}{y}, -\frac{1}{z}\right)$
\end{enumerate}

\section*{Correction 4: Calculs d'extrema}

\begin{enumerate}
    \item Point critique: $(0,0)$\\
    Hessienne: $H(0,0) = \begin{pmatrix} 2 & 0 \\ 0 & 2 \end{pmatrix}$ définie positive\\
    Minimum global en $(0,0)$ avec $f(0,0)=0$
    
    \item Points critiques: $(1,-1)$ et $(-5/3,25/9)$\\
    Hessienne: $H = \begin{pmatrix} 6x & 2 \\ 2 & 0 \end{pmatrix}$\\
    En $(1,-1)$: $H = \begin{pmatrix} 6 & 2 \\ 2 & 0 \end{pmatrix}$ (indéfinie) → point selle\\
    En $(-5/3,25/9)$: $H = \begin{pmatrix} -10 & 2 \\ 2 & 0 \end{pmatrix}$ (indéfinie) → point selle
    
    \item Point critique: $(-1,0)$\\
    Maximum local en $(-1,0)$ avec $h(-1,0) = (3\times(-1)+7)e^{-0} = 4$
\end{enumerate}

\section*{Correction 5: Intégrales multiples}

\begin{enumerate}
    \item 
    \begin{enumerate}[a)]
        \item $\int_0^1\int_0^{1-x}(x^2+y^2)dydx = \frac{1}{6}$
        \item $\int_0^1\int_0^{1-x}xy(x+y)dydx = \frac{1}{120}$
    \end{enumerate}
    
    \item 
    \begin{enumerate}[a)]
        \item $\int_0^2\int_{\frac{x+1}{2}}^{4-x}xdydx = \frac{7}{3}$
        \item $\int_0^1\int_{x^2}^{x}(x+y)dydx = \frac{3}{20}$
    \end{enumerate}
    
    \item Aire = $\int_{-1}^1\int_{x^2}^{4-x^3}dydx = \frac{52}{5}$
    
    \item $\iiint_D(x+y+z)^2dxdydz = \int_0^1\int_0^{1-x}\int_0^{1-x-y}(x+y+z)^2dzdydx = \frac{1}{60}$
    
    \item Volume = $\pi\int_{h_2}^{h_1}(R^2-z^2)dz = \pi\left[R^2(h_1-h_2) - \frac{h_1^3-h_2^3}{3}\right]$
\end{enumerate}

\section*{Correction 6: Coordonnées cylindriques et sphériques}

\begin{enumerate}
    \item[(a)] $x^2+y^2-2x \leq 0 \Rightarrow (x-1)^2+y^2 \leq 1$\\
    En polaires: $x = r\cos\theta$, $y = r\sin\theta$\\
    $(r\cos\theta-1)^2 + r^2\sin^2\theta \leq 1 \Rightarrow r^2 - 2r\cos\theta \leq 0 \Rightarrow r \leq 2\cos\theta$\\
    Avec $\theta \in [-\frac{\pi}{2}, \frac{\pi}{2}]$
    
    \item[(b)] $\iint_D \sqrt{x^2+y^2}dxdy = \int_{-\pi/2}^{\pi/2}\int_0^{2\cos\theta}r\cdot rdrd\theta = \frac{32}{9}$
\end{enumerate}