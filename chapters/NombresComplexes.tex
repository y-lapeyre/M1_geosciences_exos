\chapter{Nombres Complexes}

\section*{Exercice 1}

\begin{enumerate}[a)]
    \item  Mettre sous la forme $a+i b(a, b \in \mathbb{R})$ les nombres :
    $$
    \frac{3+6 i}{3-4 i} \quad ; \quad\left(\frac{1+i}{2-i}\right)^2+\frac{3+6 i}{3-4 i} \quad ; \quad \frac{2+5 i}{1-i}+\frac{2-5 i}{1+i}
    $$
    \item Mettez les nombres complexes suivants sous formes trigonometrique.
    $$
    1+i \quad ; \quad-1-i \quad ; \quad 1+\sqrt{3} i \quad ; \quad-1+\sqrt{3} i \quad ; \quad-6+0 i \quad ; \quad-i
    $$
    \item Soit $z=\frac{1}{2}-\frac{3}{2} i$ et $z^{\prime}=\sqrt{3}+i$. Utilisez la forme polaire pour calculer
    $$
    z z^{\prime} \quad, \quad z / z^{\prime} \quad\left(z^{\prime}\right)^5
    $$
    \item Calculer le module et l'argument de $u=\frac{\sqrt{6}-i \sqrt{2}}{2}$ et $v=1-i$. En déduire le module et l'argument de $w=\frac{u}{v}$.
    \item Déterminer le module et l'argument d'un nombre complex :
    $$
    e^{e^{i \alpha}}
    $$
    \item Calculer
    $$
    (1+i)^{2012}
    $$
\end{enumerate}


