\chapter{Probabilités}

\section*{Exercice 1:}

Soient \(\mathrm{A}, \mathrm{B}, \mathrm{C}\) trois événements. Exprimer en fonction de \(\mathrm{A}, \mathrm{B}\) et C et des opérations ensemblistes les événements suivants :

\begin{itemize}
    \item A seul se produit
    \item A et C se produisent mais pas B
    \item les trois événements se produisent
    \item deux événements au moins se produisent
    \item un événement au plus se produit
    \item aucun des trois événements ne se produit
    \item deux événements exactement se produisent
\end{itemize}

\section*{Exercice 2:}

On dispose d'un dé truqué sur lequel chaque face a une probabilité d'apparition proportionnelle au numéro qu'elle porte.

\begin{itemize}
\item Déterminer la probabilité d'apparition de chaque face de ce dé.
\item On lance deux fois ce dé. Quelle est la probabilité que la somme des points obtenus soit égale à 4 ?
\end{itemize}

\section*{Exercice 3 :}

On jette deux dés non truqués deux fois de suite.
\begin{itemize}
    \item Quelle est la probabilité d'obtenir au moins une fois un 6 ?
    \item Quelle est la probabilité que la somme des 4 faces obtenues soit 6 ?
\end{itemize}


\section*{Exercice 4 : Pièce de monnaie}

On possède une pièce de monnaie truquée de telle sorte que la probabilité d'obtenir pile soit 0,3 .

\begin{itemize}
    \item On lance 10 fois la pièce, qu'elle est la probabilité d'obtenir 3 fois pile ?
    \item On lance la pièce jusqu'à ce qu'on obtenienne pile pour la première fois. Combien effectuera-t-on de lancers en moyenne ?
\end{itemize}


\section*{Exercice 5 : Service de dépannage}

Le service de dépannage d'un grand magasin dispose d'équipes intervenant sur appel de la clientèle. Pour diverses causes, les interventions ont parfois eu lieu avec retard. On admet que les appels se produisent indépendamment les uns des autres, et que, pour chaque appel, la probabilité d'un retard est de 0,25 .

\begin{enumerate}
    \item Un client appelle le service à 4 reprises. On désigne par \(X\) la v.a. prenant pour valeurs le nombre de fois où ce client a dû subir un retard.
    \begin{itemize}
        \item Déterminer la loi de probabilité de \(X\), son espérance, sa variance.
        \item Calculer la probabilité de l'événement : "le client a au moins subi un retard".
    \end{itemize}
    \item Le nombre d'appels reçus par jour est une v.a. \(Y\) qui suit une loi de Poisson de paramètre \(m\). On note \(Z\) le nombre d'appels traités en retard.
    \begin{itemize}
        \item Exprimer la probabilité conditionnelle de \(Z=k\) sachant que \(Y=n\).
        \item En déduire la probabilité de " \(Z=k\) et \(Y=n\) "
        \item Déterminer la loi de \(Z\). On trouvera que \(Z\) suit une loi de Poisson de paramètre \(m \times 0,25\).
    \end{itemize}
    \item En 2015, le standard a reçu une succession d'appels. On note \(U\) le premier appel reçu en retard. Quelle est la loi de \(U\) ? Quelle est son espérance ?
\end{enumerate}

\section*{Correction exercice 1:}

Soient \(\mathrm{A}, \mathrm{B}, \mathrm{C}\) trois événements. Exprimer en fonction de \(\mathrm{A}, \mathrm{B}\) et C et des opérations ensemblistes les événements suivants :

\begin{itemize}
    \item A seul se produit : \(A \cap \overline{B} \cap \overline{C}\)
    \item A et C se produisent mais pas B : \(A \cap C \cap \overline{B}\)
    \item les trois événements se produisent : \(A \cap B \cap C\)
    \item deux événements au moins se produisent : \((A \cap B) \cup (A \cap C) \cup (B \cap C)\)
    \item un événement au plus se produit : \(\overline{A} \cap \overline{B} \cap \overline{C} \cup (A \cap \overline{B} \cap \overline{C}) \cup (\overline{A} \cap B \cap \overline{C}) \cup (\overline{A} \cap \overline{B} \cap C)\)
    \item aucun des trois événements ne se produit : \(\overline{A} \cap \overline{B} \cap \overline{C}\)
    \item deux événements exactement se produisent : \((A \cap B \cap \overline{C}) \cup (A \cap \overline{B} \cap C) \cup (\overline{A} \cap B \cap C)\)
\end{itemize}

\section*{Correction exercice 2:}

On dispose d'un dé truqué sur lequel chaque face a une probabilité d'apparition proportionnelle au numéro qu'elle porte.

\begin{itemize}
\item Déterminer la probabilité d'apparition de chaque face de ce dé.
\end{itemize}

Soit \(p_i = P(\text{face } i)\). On a \(p_i = k \cdot i\) avec \(k\) constante de proportionnalité.

\[\sum_{i=1}^{6} p_i = 1 \Rightarrow k(1+2+3+4+5+6) = 21k = 1 \Rightarrow k = \frac{1}{21}\]

Donc : \(p_i = \frac{i}{21}\) pour \(i = 1, 2, ..., 6\)

\begin{itemize}
\item On lance deux fois ce dé. Quelle est la probabilité que la somme des points obtenus soit égale à 4 ?
\end{itemize}

Les couples donnant une somme de 4 : (1,3), (2,2), (3,1)

\[P(\text{somme}=4) = P(1,3) + P(2,2) + P(3,1) = \frac{1}{21} \cdot \frac{3}{21} + \frac{2}{21} \cdot \frac{2}{21} + \frac{3}{21} \cdot \frac{1}{21} = \frac{3+4+3}{441} = \frac{10}{441}\]

\section*{Correction exercice 3 :}

On jette deux dés non truqués deux fois de suite.

\begin{itemize}
    \item Quelle est la probabilité d'obtenir au moins une fois un 6 ?
\end{itemize}

Probabilité de ne pas obtenir de 6 sur un lancer de deux dés : \(\frac{5}{6} \times \frac{5}{6} = \frac{25}{36}\)

Probabilité de ne jamais obtenir de 6 sur deux lancers : \((\frac{25}{36})^2 = \frac{625}{1296}\)

Probabilité d'obtenir au moins un 6 : \(1 - \frac{625}{1296} = \frac{671}{1296}\)

\begin{itemize}
    \item Quelle est la probabilité que la somme des 4 faces obtenues soit 6 ?
\end{itemize}

La somme minimale est 4 (quatre 1), la somme maximale est 24 (quatre 6).\\
La somme 6 peut être obtenue avec les combinaisons suivantes :
\begin{itemize}
    \item 1+1+1+3 (4 permutations)
    \item 1+1+2+2 (6 permutations)
    \item 1+1+1+3 : \(\frac{4}{6^4} = \frac{4}{1296}\)
    \item 1+1+2+2 : \(\frac{6}{6^4} = \frac{6}{1296}\)
\end{itemize}

Total : \(\frac{10}{1296} = \frac{5}{648}\)

\section*{Correction exercice 4 : Pièce de monnaie}

On possède une pièce de monnaie truquée de telle sorte que la probabilité d'obtenir pile soit 0,3.

\begin{itemize}
    \item On lance 10 fois la pièce, qu'elle est la probabilité d'obtenir 3 fois pile ?
\end{itemize}

Loi binomiale : \(X \sim \mathcal{B}(10; 0,3)\)

\[P(X=3) = \binom{10}{3} (0,3)^3 (0,7)^7 = 120 \times 0,027 \times 0,0823543 \approx 0,2668\]

\begin{itemize}
    \item On lance la pièce jusqu'à ce qu'on obtenienne pile pour la première fois. Combien effectuera-t-on de lancers en moyenne ?
\end{itemize}

Loi géométrique : \(Y \sim \mathcal{G}(0,3)\)

\[E[Y] = \frac{1}{p} = \frac{1}{0,3} \approx 3,33\]

\section*{Correction exercice 5 : Service de dépannage}

\begin{enumerate}
    \item Un client appelle le service à 4 reprises. On désigne par \(X\) la v.a. prenant pour valeurs le nombre de fois où ce client a dû subir un retard.
    \begin{itemize}
        \item Déterminer la loi de probabilité de \(X\), son espérance, sa variance.
        
        \(X \sim \mathcal{B}(4; 0,25)\)
        
        \(P(X=k) = \binom{4}{k} (0,25)^k (0,75)^{4-k}\)
        
        \(E[X] = 4 \times 0,25 = 1\)
        
        \(V[X] = 4 \times 0,25 \times 0,75 = 0,75\)
        
        \item Calculer la probabilité de l'événement : "le client a au moins subi un retard".
        
        \(P(X \geq 1) = 1 - P(X=0) = 1 - (0,75)^4 = 1 - 0,3164 = 0,6836\)
    \end{itemize}
    
    \item Le nombre d'appels reçus par jour est une v.a. \(Y\) qui suit une loi de Poisson de paramètre \(m\). On note \(Z\) le nombre d'appels traités en retard.
    \begin{itemize}
        \item Exprimer la probabilité conditionnelle de \(Z=k\) sachant que \(Y=n\).
        
        \(P(Z=k | Y=n) = \binom{n}{k} (0,25)^k (0,75)^{n-k}\)
        
        \item En déduire la probabilité de " \(Z=k\) et \(Y=n\) "
        
        \(P(Z=k \cap Y=n) = P(Z=k | Y=n) \cdot P(Y=n) = \binom{n}{k} (0,25)^k (0,75)^{n-k} \cdot \frac{e^{-m} m^n}{n!}\)
        
        \item Déterminer la loi de \(Z\).
        
        \(P(Z=k) = \sum_{n=k}^{\infty} P(Z=k \cap Y=n) = \sum_{n=k}^{\infty} \binom{n}{k} (0,25)^k (0,75)^{n-k} \cdot \frac{e^{-m} m^n}{n!}\)
        
        \(= \frac{e^{-m} (0,25)^k}{k!} \sum_{n=k}^{\infty} \frac{m^n (0,75)^{n-k}}{(n-k)!}\)
        
        \(= \frac{e^{-m} (0,25 m)^k}{k!} \sum_{j=0}^{\infty} \frac{(0,75 m)^j}{j!} = \frac{e^{-m} (0,25 m)^k}{k!} e^{0,75 m}\)
        
        \(= \frac{e^{-0,25 m} (0,25 m)^k}{k!}\)
        
        Donc \(Z \sim \mathcal{P}(0,25 m)\)
    \end{itemize}
    
    \item En 2015, le standard a reçu une succession d'appels. On note \(U\) le premier appel reçu en retard. Quelle est la loi de \(U\) ? Quelle est son espérance ?
    
    \(U\) suit une loi géométrique : \(P(U=k) = (0,75)^{k-1} \cdot 0,25\)
    
    \(E[U] = \frac{1}{0,25} = 4\)
\end{enumerate}

