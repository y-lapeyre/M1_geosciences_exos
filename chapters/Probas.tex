\chapter{Probabilités: dénombrement}

\section*{Exercice 1 : Pour récapituler}

Soit \(\mathbb{E}=\{1,2,3,4\}\)

\begin{enumerate}
    \item Donner le cardinal de \(\mathbb{E}\).
    \item Donner le nombre de 2 -arrangements avec répétitions possibles avec les éléments de \(\mathbb{E}\).
    \item Donner le nombre de 2 -arrangements sans répétitions possibles avec les éléments de \(\mathbb{E}\).
    \item Donner le nombre de permutations sans répétitions possibles avec les éléments de \(\mathbb{E}\).
    \item Donner le nombre de combinaisons sans répétitions possibles contenant 2 éléments de \(\mathbb{E}\).
\end{enumerate}


\section*{Exercice 2 : Cardinal d'un ensemble}

\begin{enumerate}
    \item Combien y a-t-il d'élèves dans une classe où 27 étudient l'anglais, 15 l'allemand et 9 les deux langues, sachant que chaque élève étudie au moins une langue ?
    \item Dans une classe de 31 élèves, 16 étudient l'anglais, 13 l'espagnol, 14 l'allemand, 4 l'anglais et l'espagnol, 6 l'espagnol et l'allemand, 5 l'anglais et l'allemand. Combien étudient les 3 langues ?
\end{enumerate}


\section*{Exercice 3 : Tirages de boules}

\begin{enumerate}
    \item On dispose de 8 boules numérotées et de 4 sacs numérotés. On répartit les 8 boules dans les sacs. Combien y a-t-il de répartitions possibles ? Combien y a-t-il de répartitions telles qu'aucun sac ne soit vide?
    \item Dans une urne on place \(n\) boules blanches et une seule noire. On tire simultanément \(k\) boules Déterminer d'abord le nombre de tirages sans boules noires, ensuite le nombre de tirages avec au moins une boule noire, et pour finir le nombre de tirages possibles en tout. Qu'en déduisez-vous ?
    \item Une urne contient 10 boules noires numérotées, 5 boules blanches numérotées et 3 boules rouges numérotées elles aussi. On tire simultanément 4 boules dans l'urne.
    \item Trouvez:
    \begin{itemize}
    \item Quel est le nombre de tirages possibles ?
    \item Combien de tirages contiennent au moins une boule noire ?
    \item Combien de tirages contiennent autant de boules blanches que de boules rouges ?
    \item Combien de tirages contiennent les 3 couleurs ?
    \item Combien de tirages contiennent exactement 2 couleurs ?
    \end{itemize}
\end{enumerate}


\section*{Exercice 4 : Jouons avec les mots}

\begin{enumerate}
    \item On dispose de 10 jetons de Scrabble portant les lettres de l'alphabet de A à J.
    \begin{itemize}
    \item Combien de mots de 10 lettres peut-on écrire avec ces 10 jetons ?
    \item Combien de mots de 10 lettres peut-on écrire avec ces 10 jetons où \(\mathrm{B}, \mathrm{A}\) et C apparaissent dans cet ordre et côte à côte ?
    \item Même question mais \(\mathrm{B}, \mathrm{A}\) et C ne sont pas forcément côte à côte.
    \end{itemize}
    \item Combien de mots peut-on former avec les lettres \(\mathrm{A}, \mathrm{B}, \mathrm{C}, \mathrm{D}, \mathrm{E}\) en utilisant une seule fois chaque lettre ? Combien de mots de 5 lettres commencent par A ? Combien de mots de 5 lettres commencent par A et finissent par B ? Combien de mots de 5 lettres où le A apparaît devant le B ?
\end{enumerate}

\section*{Exercice 5 : Jeux de cartes}
Dans un jeu de 32 cartes, on choisit 5 cartes au hasard. Ces 5 cartes s'appellent une "main".

\begin{enumerate}
    \item Quel est le nombre total de mains qu'on peut choisir ?
    \item Combien de mains contiennent exactement 4 as ?
    \item Combien de mains contiennent exactement 3 as et 2 rois ?
    \item Combien de mains contiennent au moins 3 rois ?
    \item Combien de mains contiennent au moins un as ?
\end{enumerate}
 
\section*{Exercice 6 : Et les gens dans tout ça ?}

Un jury est composé de 10 membres tirés au sort parmi 8 hommes et 9 femmes.

\begin{enumerate}
\item Combien de jurys différents peut-on former?
\item Combien de jurys comportant 5 hommes et 5 femmes peut-on former?
\item Monsieur X refuse de sièger avec Madame Y . Combien de jurys peut-on former dans ces conditions ?
\item Lors d'un dîner, 4 personnes disposent leur chapeau au vestiaire. Au moment du départ, les convives un peu pressés reprennent au hasard chacun un chapeau. Combien y a-t-il de possibilités au total? Combien y a-t-il de possibilités pour que personne ne reprenne son chapeau?
\end{enumerate}


\section*{Correction exercice 1 : Pour récapituler}

Soit \(\mathbb{E}=\{1,2,3,4\}\)

\begin{enumerate}
    \item Cardinal de \(\mathbb{E}\): \(|\mathbb{E}| = 4\)
    \item Nombre de 2-arrangements avec répétitions: \(4^2 = 16\)
    \item Nombre de 2-arrangements sans répétitions: \(\frac{4!}{(4-2)!} = 4 \times 3 = 12\)
    \item Nombre de permutations sans répétitions: \(4! = 24\)
    \item Nombre de combinaisons de 2 éléments: \(\binom{4}{2} = 6\)
\end{enumerate}

\section*{Correction exercice 2 : Cardinal d'un ensemble}

\begin{enumerate}
    \item Nombre d'élèves: \(27 + 15 - 9 = 33\)
    \item Soit \(x\) le nombre d'élèves étudiant les 3 langues.\\
    Par le principe d'inclusion-exclusion:\\
    \(16 + 13 + 14 - 4 - 6 - 5 + x = 31\)\\
    \(28 + x = 31\)\\
    \(x = 3\)\\
    Donc 3 élèves étudient les 3 langues.
\end{enumerate}

\section*{Correction exercice 3 : Tirages de boules}

\begin{enumerate}
    \item 
    \begin{itemize}
        \item Répartitions possibles: \(4^8 = 65536\)
        \item Répartitions sans sac vide: \(\binom{8-1}{4-1} = \binom{7}{3} = 35\)
    \end{itemize}
    
    \item 
    \begin{itemize}
        \item Tirages sans boule noire: \(\binom{n}{k}\)
        \item Tirages avec au moins une boule noire: \(\binom{n+1}{k} - \binom{n}{k}\)
        \item Tirages totaux: \(\binom{n+1}{k}\)
        \item On déduit que: \(\binom{n}{k} + \left[\binom{n+1}{k} - \binom{n}{k}\right] = \binom{n+1}{k}\)
    \end{itemize}
    
    \item Urne avec 10 noires, 5 blanches, 3 rouges (total 18 boules)
    \begin{itemize}
        \item Tirages possibles: \(\binom{18}{4} = 3060\)
        \item Tirages avec au moins une noire: \(\binom{18}{4} - \binom{8}{4} = 3060 - 70 = 2990\)
        \item Tirages avec autant de blanches que de rouges:\\
        \(\binom{10}{2}\binom{5}{1}\binom{3}{1} + \binom{10}{0}\binom{5}{2}\binom{3}{2} = 45 \times 5 \times 3 + 1 \times 10 \times 3 = 675 + 30 = 705\)
        \item Tirages avec les 3 couleurs:\\
        \(\binom{10}{2}\binom{5}{1}\binom{3}{1} + \binom{10}{1}\binom{5}{2}\binom{3}{1} + \binom{10}{1}\binom{5}{1}\binom{3}{2} = 675 + 270 + 150 = 1095\)
        \item Tirages avec exactement 2 couleurs:\\
        \(\binom{10}{4} + \binom{5}{4} + \binom{3}{4} + \binom{10}{3}\binom{5}{1} + \binom{10}{3}\binom{3}{1} + \binom{5}{3}\binom{10}{1} + \binom{5}{3}\binom{3}{1} + \binom{3}{3}\binom{10}{1} + \binom{3}{3}\binom{5}{1} = 210 + 5 + 0 + 1200 + 360 + 100 + 30 + 10 + 5 = 1920\)
    \end{itemize}
\end{enumerate}

\section*{Correction exercice 4 : Jouons avec les mots}

\begin{enumerate}
    \item Avec 10 jetons A-J:
    \begin{itemize}
        \item Mots de 10 lettres: \(10! = 3628800\)
        \item BAC groupé: considérer BAC comme une seule lettre → \(8! = 40320\)
        \item BAC dans l'ordre mais pas côte à côte: \(\frac{10!}{3!} = 604800\)
    \end{itemize}
    
    \item Avec A,B,C,D,E:
    \begin{itemize}
        \item Mots de 5 lettres: \(5! = 120\)
        \item Commençant par A: \(4! = 24\)
        \item Commençant par A et finissant par B: \(3! = 6\)
        \item A avant B: \(\frac{5!}{2} = 60\)
    \end{itemize}
\end{enumerate}

\section*{Correction exercice 5 : Jeux de cartes}

Jeu de 32 cartes, mains de 5 cartes:

\begin{enumerate}
    \item Total de mains: \(\binom{32}{5} = 201376\)
    \item 4 as: \(\binom{4}{4}\binom{28}{1} = 28\)
    \item 3 as et 2 rois: \(\binom{4}{3}\binom{4}{2} = 4 \times 6 = 24\)
    \item Au moins 3 rois: \(\binom{4}{3}\binom{28}{2} + \binom{4}{4}\binom{28}{1} = 4 \times 378 + 1 \times 28 = 1512 + 28 = 1540\)
    \item Au moins un as: \(\binom{32}{5} - \binom{28}{5} = 201376 - 98280 = 103096\)
\end{enumerate}

\section*{Correction exercice 6 : Et les gens dans tout ça ?}

\begin{enumerate}
    \item Jurys différents: \(\binom{17}{10} = 19448\)
    \item 5 hommes et 5 femmes: \(\binom{8}{5}\binom{9}{5} = 56 \times 126 = 7056\)
    \item Sans M. X et Mme Y ensemble:\\
    Total - jurys avec les deux = \(\binom{17}{10} - \binom{15}{8} = 19448 - 6435 = 13013\)
    \item Dérangements:
    \begin{itemize}
        \item Total: \(4! = 24\)
        \item Personne ne reprend son chapeau: \(!4 = 9\)
    \end{itemize}
\end{enumerate}
