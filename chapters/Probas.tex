\chapter{Probabilités: dénombrement}

\section*{Exercice 1 : Pour récapituler}

Soit \(\mathbb{E}=\{1,2,3,4\}\)
1. Donner le cardinal de \(\mathbb{E}\).
2. Donner le nombre de 2 -arrangements avec répétitions possibles avec les éléments de \(\mathbb{E}\).
3. Donner le nombre de 2 -arrangements sans répétitions possibles avec les éléments de \(\mathbb{E}\).
4. Donner le nombre de permutations sans répétitions possibles avec les éléments de \(\mathbb{E}\).
5. Donner le nombre de combinaisons sans répétitions possibles contenant 2 éléments de \(\mathbb{E}\).

\section*{Exercice 2 : Cardinal d'un ensemble}
1. Combien y a-t-il d'élèves dans une classe où 27 étudient l'anglais, 15 l'allemand et 9 les deux langues, sachant que chaque élève étudie au moins une langue ?
2. Dans une classe de 31 élèves, 16 étudient l'anglais, 13 l'espagnol, 14 l'allemand, 4 l'anglais et l'espagnol, 6 l'espagnol et l'allemand, 5 l'anglais et l'allemand. Combien étudient les 3 langues ?

\section*{Exercice 3 : Tirages de boules}
1. On dispose de 8 boules numérotées et de 4 sacs numérotés. On répartit les 8 boules dans les sacs. Combien y a-t-il de répartitions possibles ? Combien y a-t-il de répartitions telles qu'aucun sac ne soit vide?
2. Dans une urne on place \(n\) boules blanches et une seule noire. On tire simultanément \(k\) boules Déterminer d'abord le nombre de tirages sans boules noires, ensuite le nombre de tirages avec au moins une boule noire, et pour finir le nombre de tirages possibles en tout. Qu'en déduisez-vous ?
3. Une urne contient 10 boules noires numérotées, 5 boules blanches numérotées et 3 boules rouges numérotées elles aussi. On tire simultanément 4 boules dans l'urne.
- Quel est le nombre de tirages possibles ?
- Combien de tirages contiennent au moins une boule noire ?
- Combien de tirages contiennent autant de boules blanches que de boules rouges ?
- Combien de tirages contiennent les 3 couleurs ?
- Combien de tirages contiennent exactement 2 couleurs ?

\section*{Exercice 4 : Jouons avec les mots}
1. On dispose de 10 jetons de Scrabble portant les lettres de l'alphabet de A à J.
- Combien de mots de 10 lettres peut-on écrire avec ces 10 jetons ?
- Combien de mots de 10 lettres peut-on écrire avec ces 10 jetons où \(\mathrm{B}, \mathrm{A}\) et C apparaissent dans cet ordre et côte à côte ?
- Même question mais \(\mathrm{B}, \mathrm{A}\) et C ne sont pas forcément côte à côte.
2. Combien de mots peut-on former avec les lettres \(\mathrm{A}, \mathrm{B}, \mathrm{C}, \mathrm{D}, \mathrm{E}\) en utilisant une seule fois chaque lettre ? Combien de mots de 5 lettres commencent par A ? Combien de mots de 5 lettres commencent par A et finissent par B ? Combien de mots de 5 lettres où le A apparaît devant le B ?

\section*{Exercice 5 : Jeux de cartes}

Dans un jeu de 32 cartes, on choisit 5 cartes au hasard. Ces 5 cartes s'appellent une "main".
1. Quel est le nombre total de mains qu'on peut choisir ?
2. Combien de mains contiennent exactement 4 as ?
3. Combien de mains contiennent exactement 3 as et 2 rois ?
4. Combien de mains contiennent au moins 3 rois ?
5. Combien de mains contiennent au moins un as ?

\section*{Exercice 6 : Et les gens dans tout ça ?}

Un jury est composé de 10 membres tirés au sort parmi 8 hommes et 9 femmes.
1. Combien de jurys différents peut-on former?
2. Combien de jurys comportant 5 hommes et 5 femmes peut-on former?
3. Monsieur X refuse de sièger avec Madame Y . Combien de jurys peut-on former dans ces conditions ?
4. Lors d'un dîner, 4 personnes disposent leur chapeau au vestiaire. Au moment du départ, les convives un peu pressés reprennent au hasard chacun un chapeau. Combien y a-t-il de possibilités au total? Combien y a-t-il de possibilités pour que personne ne reprenne son chapeau?