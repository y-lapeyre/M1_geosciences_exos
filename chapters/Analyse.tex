\chapter{Analyse}


\newcommand{\E}[1]{\cdot 10^{#1}}
\newcommand{\V}[1]{\mathbf{#1}}
\newcommand{\R}{\mathbb{R}}

\newcommand{\etheta}{\boldsymbol{\hat\theta}}
\newcommand{\ephi}{\boldsymbol{\hat\phi}}

\newcommand{\grad}{\boldsymbol{\nabla}}
\newcommand{\dive}{\nabla\cdot}
\newcommand{\curl}{\nabla\times}
\newcommand{\lapl}{\nabla^2}

\newcommand{\pfrac}[3][]{\dfrac{\partial^{#1} #2}{\partial #3^{#1}}}
\newcommand{\ppfrac}[3][]{\pfrac[#1]{}{#3}\left({#2}\right)}
\newcommand{\ffrac}[3][]{\dfrac{\mathrm d^{#1} #2}{\mathrm d#3^{#1}}}
\newcommand{\fffrac}[3][]{\ffrac[#1]{}{#3}\left({#2}\right)}
\newcommand{\lfrac}[2]{\dfrac{D #1}{D #2}}

\newcommand{\meanh}[1]{\langle{#1}\rangle_h}
\newcommand{\Od}[2][]{\mathcal{O}(\Delta #2^{#1})}

%\numberwithin{equation}{section}


\section*{Exercice 1}
Donner le domaine de définition, le domaine de dérivabilité et la
dérivée première et seconde des fonctions suivantes.

Tracer l'allure de leurs courbes représentatives fonctions en recherchant les
limites intéressantes, le signe de la dérivée première et seconde, les
points de discontinuité de la fonction ou de ses dérivées, etc.

\[\begin{aligned}
    f_1\colon x&\mapsto\frac{3x^2+5x+2}{3x+2} &
    f_2\colon x&\mapsto\sqrt{x}\left ( 1-\frac{1}{x}\right )  &
    f_3\colon x&\mapsto3x^4-2x^3+5x-6\\
    f_4\colon x&\mapsto|x^2-x-1| &
    f_5\colon x&\mapsto(3-2x)^2(3+2x^3) &
    f_6\colon x&\mapsto\frac{3-2x}{3+2x}\\
    f_7\colon x&\mapsto\sqrt{4-2x^2} &
    f_8\colon x&\mapsto\cos(2x^2+3) &
    f_9\colon x&\mapsto x^2 e^{2x+1}\\
    f_{10}\colon x&\mapsto\left (3x-1 \right ) ^4 &
    f_{11}\colon x&\mapsto\ln \left ( 1+e^{-x}\right ) &
    f_{12}\colon x&\mapsto x-\ln x-\frac{1}{x}\\
\end{aligned}\]

\section*{Exercice 2}
Soit $\mathcal{P}$ la parabole étant la courbe représentative de $f\colon
x\mapsto x^2-3x+1$. Calculer les coordonnées de son sommet $S$.

\section*{Exercice 3}
On consid\`ere la fonction $f$ définie par 
\begin{equation}
  f(x)=\frac{x^2-3x+6}{x-2}.
\end{equation}
Etudier cette fonction (domaine de définition, limites, tableau de variation). 

Déterminer trois réels $a$, $b$ et $c$ tels que
\begin{equation}
  f(x)=ax+b+\frac{c}{x-2}.
\end{equation}
En déduire que la droite $\Delta$ d'équation $y=x-1$ est une asymptote oblique \`a la courbe
représentative de  $f$.

\section*{Exercice 4}
Le profil gravimétrique d'une anomalie sphérique de rayon $r$, de densité
$\Delta \rho$ et située à la profondeur $h$ est donné par la
fonction $\Delta g$ suivante :

\begin{equation}
  \Delta g(x)=G\frac{4}{3} \pi \Delta \rho \frac{r^3 h }{(h^2+(x-3)^2)^{3/2}}.
\end{equation}

Etudier la fonction et dessiner sa courbe représentative. Donner le maximum
$\Delta g_0$ de la fonction, et réécrivez la sous la forme $\Delta g/\Delta
g_0$.

\section*{Exercice 5}

\begin{enumerate}
    \item Calculer la série de Taylor en $0$ de la fonction exponentielle.
        Sachant que la fonction exponentielle est entière, en déduire une
        expression de $e$.
    \item Calculer la série de Taylor en $0$ de $x\mapsto ax^3+bx^2+cx+d$.
    \item Calculer le $DL_n$ en $0$ de $x\mapsto \dfrac{1}{1-x}$.
    \item Calculer le $DL_3$ en $0$ de $x\mapsto e^{\sin x}$.
    \item Calculer le $DL_3$ en $2$ de $x\mapsto \ln (2-\sqrt{x-1})$.
    \item Calculer ${\displaystyle\lim_{x\to 0}\frac{\cos x-e^{-x^2/2}}{x^4}}$.
    \item Donner le signe de $\dfrac{x+2}{2x}\ln(1+x)-1$ au voisinage de 0.
\end{enumerate}


