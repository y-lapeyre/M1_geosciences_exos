\chapter{Analyse}


\newcommand{\E}[1]{\cdot 10^{#1}}
\newcommand{\V}[1]{\mathbf{#1}}
\newcommand{\R}{\mathbb{R}}

\newcommand{\etheta}{\boldsymbol{\hat\theta}}
\newcommand{\ephi}{\boldsymbol{\hat\phi}}

\newcommand{\grad}{\boldsymbol{\nabla}}
\newcommand{\dive}{\nabla\cdot}
\newcommand{\curl}{\nabla\times}
\newcommand{\lapl}{\nabla^2}

\newcommand{\pfrac}[3][]{\dfrac{\partial^{#1} #2}{\partial #3^{#1}}}
\newcommand{\ppfrac}[3][]{\pfrac[#1]{}{#3}\left({#2}\right)}
\newcommand{\ffrac}[3][]{\dfrac{\mathrm d^{#1} #2}{\mathrm d#3^{#1}}}
\newcommand{\fffrac}[3][]{\ffrac[#1]{}{#3}\left({#2}\right)}
\newcommand{\lfrac}[2]{\dfrac{D #1}{D #2}}

\newcommand{\meanh}[1]{\langle{#1}\rangle_h}
\newcommand{\Od}[2][]{\mathcal{O}(\Delta #2^{#1})}

%\numberwithin{equation}{section}


\section*{Exercice 1}
Donner le domaine de définition, le domaine de dérivabilité et la
dérivée première et seconde des fonctions suivantes.

Tracer l'allure de leurs courbes représentatives fonctions en recherchant les
limites intéressantes, le signe de la dérivée première et seconde, les
points de discontinuité de la fonction ou de ses dérivées, etc.

\[\begin{aligned}
    f_1\colon x&\mapsto\frac{3x^2+5x+2}{3x+2} &
    f_2\colon x&\mapsto\sqrt{x}\left ( 1-\frac{1}{x}\right )  &
    f_3\colon x&\mapsto3x^4-2x^3+5x-6\\
    f_4\colon x&\mapsto|x^2-x-1| &
    f_5\colon x&\mapsto(3-2x)^2(3+2x^3) &
    f_6\colon x&\mapsto\frac{3-2x}{3+2x}\\
    f_7\colon x&\mapsto\sqrt{4-2x^2} &
    f_8\colon x&\mapsto\cos(2x^2+3) &
    f_9\colon x&\mapsto x^2 e^{2x+1}\\
    f_{10}\colon x&\mapsto\left (3x-1 \right ) ^4 &
    f_{11}\colon x&\mapsto\ln \left ( 1+e^{-x}\right ) &
    f_{12}\colon x&\mapsto x-\ln x-\frac{1}{x}\\
\end{aligned}\]

\section*{Exercice 2}
Soit $\mathcal{P}$ la parabole étant la courbe représentative de $f\colon
x\mapsto x^2-3x+1$. Calculer les coordonnées de son sommet $S$.

\section*{Exercice 3}
On consid\`ere la fonction $f$ définie par 
\begin{equation}
  f(x)=\frac{x^2-3x+6}{x-2}.
\end{equation}
Etudier cette fonction (domaine de définition, limites, tableau de variation). 

Déterminer trois réels $a$, $b$ et $c$ tels que
\begin{equation}
  f(x)=ax+b+\frac{c}{x-2}.
\end{equation}
En déduire que la droite $\Delta$ d'équation $y=x-1$ est une asymptote oblique \`a la courbe
représentative de  $f$.

\section*{Exercice 4}
Le profil gravimétrique d'une anomalie sphérique de rayon $r$, de densité
$\Delta \rho$ et située à la profondeur $h$ est donné par la
fonction $\Delta g$ suivante :

\begin{equation}
  \Delta g(x)=G\frac{4}{3} \pi \Delta \rho \frac{r^3 h }{(h^2+(x-3)^2)^{3/2}}.
\end{equation}

Etudier la fonction et dessiner sa courbe représentative. Donner le maximum
$\Delta g_0$ de la fonction, et réécrivez la sous la forme $\Delta g/\Delta
g_0$.

\section*{Exercice 5}

\begin{enumerate}
    \item Calculer la série de Taylor en $0$ de la fonction exponentielle.
        Sachant que la fonction exponentielle est entière, en déduire une
        expression de $e$.
    \item Calculer la série de Taylor en $0$ de $x\mapsto ax^3+bx^2+cx+d$.
    \item Calculer le $DL_n$ en $0$ de $x\mapsto \dfrac{1}{1-x}$.
    \item Calculer le $DL_3$ en $0$ de $x\mapsto e^{\sin x}$.
    \item Calculer le $DL_3$ en $2$ de $x\mapsto \ln (2-\sqrt{x-1})$.
    \item Calculer ${\displaystyle\lim_{x\to 0}\frac{\cos x-e^{-x^2/2}}{x^4}}$.
    \item Donner le signe de $\dfrac{x+2}{2x}\ln(1+x)-1$ au voisinage de 0.
\end{enumerate}


\section*{Exercice 6: Développements limités}
\subsection*{Faciles}
Calculer les développements limités suivants :
\begin{enumerate}
  \item $\frac{1}{1-x}-e^x$ à l'ordre 3 en 0
  \item $\sqrt{1-x}+\sqrt{1+x}$ à l'ordre 4 en 0
  \item $\sin x \cos (2 x)$ à l'ordre 6 en 0
  \item $\cos (x) \ln (1+x)$ à l'ordre 4 en 0
  \item $\left(x^3+1\right) \sqrt{1-x}$ à l'ordre 3 en 0
  \item $(\ln (1+x))^2$ à l'ordre 4 en 0
\end{enumerate}

\subsection*{Un peu plus dur (mais pas trop)}
Calculer les développements limités suivants :

\begin{enumerate}
  \item $\frac{1}{1+x+x^2}$ à l'ordre 4 en 0
  \item $\tan (x)$ à l'ordre 5 en 0
  \item $\frac{\sin x-1}{\cos x+1}$ à l'ordre 2 en 0
  \item $\frac{\ln (1+x)}{\sin x}$ à l'ordre 3 en 0 .
  \item $\ln \left(\frac{\sin x}{x}\right)$ à l'ordre 4 en 0
  \item $\exp (\sin x)$ à l'ordre 4 en 0
  \item $(\cos x)^{\sin x}$ à l'ordre 5 en 0
  \item $x(\cosh x)^{\frac{1}{x}}$ à l'ordre 4 en 0 .
\end{enumerate}

Pour plus d'exos corrigés sur les DL: \url{https://www.bibmath.net/ressources/index.php?action=affiche&quoi=bde/analyse/unevariable/dl&type=fexo}

\section*{Corrigé exo 5}
1)

La fonction exponentielle est entière (développable en série entière sur $\mathbb{R}$ avec un rayon de convergence infini). Sa série de Taylor en 0 est donnée par :

\[
e^x = \sum_{n=0}^{+\infty} \frac{f^{(n)}(0)}{n!} x^n
\]
Or, pour tout $n \in \mathbb{N}$, $f^{(n)}(x) = e^x$, donc $f^{(n)}(0) = 1$. Ainsi :
\[
e^x = \sum_{n=0}^{+\infty} \frac{x^n}{n!} = 1 + x + \frac{x^2}{2!} + \frac{x^3}{3!} + \cdots
\]

Donc 
\[
e = e^1 = \sum_{n=0}^{+\infty} \frac{1}{n!} = 1 + 1 + \frac{1}{2!} + \frac{1}{3!} + \cdots
\]

2)

La fonction est un polynôme de degré 3.

On a :
\begin{align*}
P(0) &= d \\
P'(x) &= 3a x^2 + 2b x + c \quad \Rightarrow \quad P'(0) = c \\
P''(x) &= 6a x + 2b \quad \Rightarrow \quad P''(0) = 2b \\
P'''(x) &= 6a \quad \Rightarrow \quad P'''(0) = 6a \\
P^{(k)}(x) &= 0 \quad \text{pour } k \geq 4
\end{align*}

Ainsi, la série de Taylor est :
\[
P(x) = P(0) + P'(0)x + \frac{P''(0)}{2!} x^2 + \frac{P'''(0)}{3!} x^3 = d + c x + \frac{2b}{2} x^2 + \frac{6a}{6} x^3 = d + c x + b x^2 + a x^3
\]

3)
On utilise le développement limité de $(1+u)^\alpha$ avec $u = -x$ et $\alpha = \frac{1}{2}$.

\[
(1+u)^\alpha = 1 + \alpha u + \frac{\alpha(\alpha-1)}{2!} u^2 + \cdots + \frac{\alpha(\alpha-1)\cdots(\alpha-n+1)}{n!} u^n + o(u^n)
\]

Ici :
\[
\sqrt{1-x} = (1 + (-x))^{1/2} = 1 + \frac{1}{2} (-x) + \frac{\frac{1}{2} \left(-\frac{1}{2}\right)}{2!} (-x)^2 + \cdots + \frac{\frac{1}{2} \left(-\frac{1}{2}\right) \cdots \left(\frac{1}{2} - n + 1\right)}{n!} (-x)^n + o(x^n)
\]

Simplifions les termes :
\begin{align*}
T_0 &= 1 \\
T_1 &= \frac{1}{2} (-x) = -\frac{1}{2} x \\
T_2 &= \frac{\frac{1}{2} \cdot \left(-\frac{1}{2}\right)}{2} x^2 = \frac{-1/4}{2} x^2 = -\frac{1}{8} x^2 \\
T_3 &= \frac{\frac{1}{2} \cdot \left(-\frac{1}{2}\right) \cdot \left(-\frac{3}{2}\right)}{6} (-x)^3 = \frac{(-1/2)(-1/2)(-3/2)}{6} (-x^3) = \frac{(-3/8)}{6} (-x^3) = \frac{-3/8 \cdot (-1)}{6} x^3 = \frac{3/8}{6} x^3 = \frac{1}{16} x^3
\end{align*}
Attention : le signe alterne. On peut écrire le terme général :
\[
T_k = \binom{1/2}{k} (-x)^k = \binom{1/2}{k} (-1)^k x^k
\]
où on a noté le coefficient binomial "k parmi n": $\binom{n}{k} = \frac{n(n-1)\cdots(n-k+1)}{k!}$.

Ainsi :
\[
\sqrt{1-x} = \sum_{k=0}^n \binom{1/2}{k} (-1)^k x^k + o(x^n)
\]

4)
On compose les développements limités.

D'abord, $\sin x = x - \frac{x^3}{6} + o(x^3)$.

Ensuite, $e^u = 1 + u + \frac{u^2}{2} + \frac{u^3}{6} + o(u^3)$.

On pose $u = \sin x = x - \frac{x^3}{6} + o(x^3)$.

Alors :
\begin{align*}
e^{\sin x} &= 1 + \left(x - \frac{x^3}{6}\right) + \frac{1}{2} \left(x - \frac{x^3}{6}\right)^2 + \frac{1}{6} \left(x - \frac{x^3}{6}\right)^3 + o(x^3) \\
&= 1 + x - \frac{x^3}{6} + \frac{1}{2} \left(x^2 - 2 \cdot x \cdot \frac{x^3}{6} + \left(\frac{x^3}{6}\right)^2\right) + \frac{1}{6} \left(x^3 + 3 \cdot x^2 \cdot \left(-\frac{x^3}{6}\right) + \cdots\right) + o(x^3) \\
&= 1 + x - \frac{x^3}{6} + \frac{1}{2} \left(x^2 - \frac{x^4}{3} + \frac{x^6}{36}\right) + \frac{1}{6} \left(x^3 - \frac{x^5}{2} + \cdots\right) + o(x^3)
\end{align*}
On ne garde que les termes de degré $\leq 3$ :
\begin{align*}
&= 1 + x - \frac{x^3}{6} + \frac{1}{2} x^2 + \frac{1}{6} x^3 + o(x^3) \\
&= 1 + x + \frac{1}{2} x^2 + \left(-\frac{1}{6} + \frac{1}{6}\right) x^3 + o(x^3) \\
&= 1 + x + \frac{1}{2} x^2 + o(x^3)
\end{align*}

Ainsi :
\[
e^{\sin x} = 1 + x + \frac{x^2}{2} + o(x^3)
\]

5)

On pose $x = 2 + h$, avec $h \to 0$. On cherche un DL$_3$ en $h$.

Soit $f(x) = \ln(2 - \sqrt{x-1})$.

Avec $x = 2 + h$, alors $x - 1 = 1 + h$, donc $\sqrt{x-1} = \sqrt{1+h}$.

On a donc :
\[
f(2+h) = \ln(2 - \sqrt{1+h})
\]

Développons $\sqrt{1+h}$ à l'ordre 3 en 0 :
\[
\sqrt{1+h} = 1 + \frac{1}{2} h - \frac{1}{8} h^2 + \frac{1}{16} h^3 + o(h^3)
\]

Alors :
\[
2 - \sqrt{1+h} = 2 - \left(1 + \frac{1}{2} h - \frac{1}{8} h^2 + \frac{1}{16} h^3\right) + o(h^3) = 1 - \frac{1}{2} h + \frac{1}{8} h^2 - \frac{1}{16} h^3 + o(h^3)
\]

On écrit :
\[
2 - \sqrt{1+h} = 1 + u \quad \text{avec} \quad u = -\frac{1}{2} h + \frac{1}{8} h^2 - \frac{1}{16} h^3 + o(h^3)
\]

On a alors :
\[
\ln(1+u) = u - \frac{u^2}{2} + \frac{u^3}{3} + o(u^3)
\]

Calculons $u$ et ses puissances :
\begin{align*}
u &= -\frac{1}{2} h + \frac{1}{8} h^2 - \frac{1}{16} h^3 + o(h^3) \\
u^2 &= \left(-\frac{1}{2} h + \frac{1}{8} h^2\right)^2 + o(h^3) = \frac{1}{4} h^2 - 2 \cdot \frac{1}{2} \cdot \frac{1}{8} h^3 + o(h^3) = \frac{1}{4} h^2 - \frac{1}{8} h^3 + o(h^3) \\
u^3 &= \left(-\frac{1}{2} h\right)^3 + o(h^3) = -\frac{1}{8} h^3 + o(h^3)
\end{align*}

Ainsi :
\begin{align*}
\ln(1+u) &= \left(-\frac{1}{2} h + \frac{1}{8} h^2 - \frac{1}{16} h^3\right) - \frac{1}{2} \left(\frac{1}{4} h^2 - \frac{1}{8} h^3\right) + \frac{1}{3} \left(-\frac{1}{8} h^3\right) + o(h^3) \\
&= -\frac{1}{2} h + \frac{1}{8} h^2 - \frac{1}{16} h^3 - \frac{1}{8} h^2 + \frac{1}{16} h^3 - \frac{1}{24} h^3 + o(h^3) \\
&= -\frac{1}{2} h + \left(\frac{1}{8} - \frac{1}{8}\right) h^2 + \left(-\frac{1}{16} + \frac{1}{16} - \frac{1}{24}\right) h^3 + o(h^3) \\
&= -\frac{1}{2} h - \frac{1}{24} h^3 + o(h^3)
\end{align*}

Donc :
\[
f(2+h) = -\frac{1}{2} h - \frac{1}{24} h^3 + o(h^3)
\]
Et en revenant à $x$ : $h = x-2$, donc :
\[
\ln(2 - \sqrt{x-1}) = -\frac{1}{2} (x-2) - \frac{1}{24} (x-2)^3 + o((x-2)^3)
\]

6)

On utilise les développements limités à l'ordre 4.

D'abord :
\[
\cos x = 1 - \frac{x^2}{2} + \frac{x^4}{24} + o(x^4)
\]

Ensuite, pour $e^{-x^2/2}$ : on pose $u = -x^2/2$, alors $e^u = 1 + u + \frac{u^2}{2} + \frac{u^3}{6} + o(u^3)$. Comme $u = O(x^2)$, il faut aller à l'ordre 4 en $x$ : donc $e^u$ jusqu'à l'ordre $u^2$ suffit car $u^2$ est d'ordre $x^4$.

Ainsi :
\[
e^{-x^2/2} = 1 + \left(-\frac{x^2}{2}\right) + \frac{1}{2} \left(-\frac{x^2}{2}\right)^2 + o(x^4) = 1 - \frac{x^2}{2} + \frac{1}{2} \cdot \frac{x^4}{4} + o(x^4) = 1 - \frac{x^2}{2} + \frac{x^4}{8} + o(x^4)
\]

Alors :
\[
\cos x - e^{-x^2/2} = \left(1 - \frac{x^2}{2} + \frac{x^4}{24}\right) - \left(1 - \frac{x^2}{2} + \frac{x^4}{8}\right) + o(x^4) = \frac{x^4}{24} - \frac{x^4}{8} + o(x^4) = \left(\frac{1}{24} - \frac{3}{24}\right) x^4 + o(x^4) = -\frac{2}{24} x^4 + o(x^4) = -\frac{1}{12} x^4 + o(x^4)
\]

Donc :
\[
\frac{\cos x - e^{-x^2/2}}{x^4} = -\frac{1}{12} + o(1) \to -\frac{1}{12} \quad \text{quand } x \to 0
\]

Ainsi :
\[
\lim_{x \to 0} \frac{\cos x - e^{-x^2/2}}{x^4} = -\frac{1}{12}
\]

7)

On étudie la fonction :
\[
f(x) = \frac{x+2}{2x} \ln(1+x) - 1
\]
pour $x$ au voisinage de 0 (avec $x \neq 0$).

Simplifions :
\[
f(x) = \frac{(x+2) \ln(1+x) - 2x}{2x}
\]

On étudie le numérateur : $N(x) = (x+2) \ln(1+x) - 2x$.

Développons $\ln(1+x) = x - \frac{x^2}{2} + \frac{x^3}{3} + o(x^3)$.

Alors :
\begin{align*}
N(x) &= (x+2) \left(x - \frac{x^2}{2} + \frac{x^3}{3}\right) - 2x + o(x^3) \\
&= (x+2)x - (x+2)\frac{x^2}{2} + (x+2)\frac{x^3}{3} - 2x + o(x^3) \\
&= x^2 + 2x - \frac{x^3}{2} - x^2 + \frac{x^4}{3} + \frac{2x^3}{3} - 2x + o(x^3) \\
&= (2x - 2x) + (x^2 - x^2) + \left(-\frac{1}{2} x^3 + \frac{2}{3} x^3\right) + o(x^3) \\
&= \left(-\frac{3}{6} + \frac{4}{6}\right) x^3 + o(x^3) = \frac{1}{6} x^3 + o(x^3)
\end{align*}

Ainsi :
\[
N(x) = \frac{1}{6} x^3 + o(x^3)
\]

Et le dénominateur est $D(x) = 2x$.

Donc :
\[
f(x) = \frac{\frac{1}{6} x^3 + o(x^3)}{2x} = \frac{1}{12} x^2 + o(x^2)
\]

Ainsi, au voisinage de 0, $f(x) \sim \frac{1}{12} x^2$, qui est positif pour $x \neq 0$.

Donc, au voisinage de 0 (pour $x$ proche de 0 mais non nul), $f(x) > 0$.



